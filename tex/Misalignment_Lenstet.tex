\subsection{Misalignment of Lenslet/Horn Array}

\paragraph{Description:}
With the lenslet coupled antenna technology, there is a possibility that the lenslets will not be aligned with an antennae. This can occur with individual pixels or the entire wafer. In the case of individual pixels, this offset would be in random directions across the wafer and thus average out across the array. In the case of entire of wafer, misalignment can be caused cause by how well the wafer and lenslet array are clamped together and thermal contraction between the wafer holder and the wafer. An entire wafer offset would create a systematic pointing error.

\paragraph{Plan to model and/or measure:}
We can create a model in HFSS to simulate the effect on beam parameters from a systematic misalignment between a lenslet and an antenna. We can then use Zemax/Grasp to propagate the beam to the sky and estimate how much this effect contributes to the pointing error. If the location of the wafer and its orientation relative to a beam mapper, the pointing offset could be measured in lab, but this could be difficult in practice. In the field, the pointing offset can be characterized with point source observations (either on a tower/drone in the farfield or celestial) as described in the previous sections as part of the full pointing model.

PB2 models of this effect show a linear relationship between the pointing error and the offset between the lenslet and antenna (see Figure~\ref{poitingoffsetFromWaferslipped}). The direction of thepointing offset is in the same direction as the wafer slip. 
  
\begin{figure}
\centering
\includegraphics[width=3.25in]{figures/pointingOffset_waferslipped.png}
\caption{Pointing error versus an offset between a sinuous antenna and a lenslet for PB2. The pointing offset roughly linear with the offset between the two.}
\label{poitingoffsetFromWaferslipped}
\end{figure}


We can model this effect well, but need to model it, making its SRF a 3.

\paragraph{Uncertainty/Range:}
We can minimize the uncertainty from this effect by understanding the effects that can cause a wafer to slip like thermal contraction from the wafer holder and wafer or outside vibrations and setting contraints on wafer slippage. The range of the uncertainty also varies with the size of lenslets and antenna. In the PB2 case, we set the tolerance level at 20 $\mu m$. \textbf{Do we have some tolerance numbers for SO?}

\paragraph{Parameterization:}
We parameterize this effect as a pointing offset.
