\subsection{Polarization Angle Frequency Dependence of HWP}

\paragraph{Description:}
%Description of systematic effect, including relevant equations and
%parameterization for TWGs. Note that each variable in each equation should be
%defined. This should include where we expect to get the value of this variable
%from (TWG, literature, etc.)

We define polarization angle as the angle betweeen the polarization of the incident light and the transmitted angle induced by the HWP. This polarization angle is frequency-dependent and is defined by the optical properties of the HWP.

One strategy to mitigate this systematic effect is to use an additional stationary HWP behind the rotating one to counteract the frequency dependence while maintaining broadband polarization modulation as proposed in \cite{Matsumura14}. 

\paragraph{Plan to model and/or measure:}
%Plan to model/measure effect. Use TRLs to describe how well we understand/can model the effect.
Using the Muller Matrix paramaterization is sufficient to model this effect, although it also could also be modeled using a full transmission line model. The transmission line model could more accuratly capture the behavior since it takes multiple reflections between the various layers into account. For a silicon metamaterial HWP, the phase shifter and its AR coating can be modeled in HFSS, and the frequency dependence can then be derived from the HFSS simulations. This effect could also be measured with a Vector Network Analyzer (VNA) setup.

\paragraph{Uncertainty/Range:}
%This section should include the uncertainty of
%known parameters and/or the expected range of parameters for consideration
Discussion in \cite{Matsumura09} shows that this can be up to about 20 degrees for some HWPs. Calculations within SO are forthcoming. \textbf{Say something quantatative about the SO HWPs. Make sure to include frequency dependence.}

\paragraph{Parameterization:}
%This section should include the parameterization of figures of
%merit and the output to the SWGs.

As discussed in \cite{Matsumura09}, a Muller Matrix representation of the HWP stack can give the frequency dependence of the polarization angle effectively.


