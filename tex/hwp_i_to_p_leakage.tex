% !TEX root =  ../syst_master.tex 
\subsection{HWP $I \rightarrow P$ Leakage Coming from Differential Optical Properties of a HWP} 

\paragraph{Description:}
Though light polarized by the HWP primarily contributes to the 2f component, a small amount of it can be modulated 
up to 4f.
This occurs when reflection induced polarization caused by irregularies in the AR coating is rotated by the HWP\cite{Essinger-Hileman2013, Essinger-Hileman2016, ABS_HWP}. 
Because it is reflected light being leaked into the 4f signal, this can come from both upstream and downstream from the HWP.

\paragraph{Plan to model and/or measure:}
The leakage can be modelled with a transfer-matrix model of the HWP, and end-to-end pipeline simulations can be used to measure the total leaked polarization power.
To measure the leakage, one can measure 4f as a function of sky I/PWV (Precipitable Water Vapour), or alternatively make maps of known unpolarised sources (like Jupiter) in the instrument Q/U frame
and measure the polarization. However both these method will characterize the overall $I \rightarrow P$ upstream of the HWP rather than just the HWP $I \rightarrow P$.
Any detector non-linearity will also Any detector non linear behavior will also contribute to this systematic effect (see Sec. \ref{}). \textbf{missing section reference}


\paragraph{Parameterization:}
The leakage coefficients are the elements $M_{12}$ and $M_{34}$ of the HWP mueller matrix M. These coefficients can be expanded into a beam Gauss-Hermite functions, 
see Equation 8 of \cite{Essinger-Hileman2016}. 

\paragraph{Uncertainty/Range:}
The effect is estimated to be small, in ABS, monopole leakage is 0.01\% and higher order leakage < 0.07\% \cite{Essinger-Hileman2016}.
Range comes from differential emissivity and transmission of the sapphire, reflection of radiation from the detectors to the HWP and back to detectors, 
misalignment of HWP axes. $I \rightarrow P$ leakage effect increases at non-normal incidence.
