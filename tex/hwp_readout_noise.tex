
\subsection{HWP Readout 1/f Noise}


\paragraph{Description:}

HWPs nominally separate the sky polarized signal from atmospheric and readout low frequency power by modulation at 4$f$, however due to the presence of a HWP Synchronous Signal (HWPSS) and temperature to polarization leakage, real demodulation methods will mix low frequency readout noise and gain drifts into demodulated polarization channels. 
Here we discuss readout 1/f power uncorrelated with atmospheric intensity. Generically this has multiplicative (small signal gain that acts on the 4f line) and additive (fluctuations on bias power impacting the 0f) effects . Following Eq. 4.6) - 4.8) in Takakura et al. (2017), let $\Delta(t)$ be an additive 1/f mode and $\delta g(t)$ be a multiplicative readout gain instability.

$$
d_{m}(t) = (\delta I (t) + \Delta (t)) \times (1 + g_{1}d_{m}(t) + \delta g(t)) + …
$$
$$
d_{d}^{sub}(t) = Q + iU + A_{0}^{4} \delta g(t) - \lambda_{4} \Delta (t) + …
$$

\paragraph{Plan to model and/or measure:}

Upper limits on the effect can be determined by measuring the thermal stability of the readout electronics enclosure in conjunction with the thermal coefficients of elements in the gain path in readout electronics, however, in practice the overall system thermal coefficient depends heavily on component matching which is difficult to estimate without a direct end to end measurement.

This end-to-end measurement can be done using overbiased detectors or resistor channels. 

\paragraph{Uncertainty/Range:}

This has the potential to be a very significant limitation in low frequency statistical sensitivity. This effect can be common mode across the focal plane and does not integrate down with detector count. In principle it is possible to monitor and project out the ADC / DAC gain modes in an FDM system by either accurately tracking the temperature in the electronics enclosure or by creating an end to end gain monitor by outputting low frequency tones fed back into thermally stable ADCs.


\paragraph{Parameterization:}

This can be described using the fractional gain instability $\delta g(t)$ and additive readout current $\Delta (t)$ referred to sky temperature. The effect should be common mode across the focal plane.
