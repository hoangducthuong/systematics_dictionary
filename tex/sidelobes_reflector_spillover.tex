\subsection{Reflector Spillover}

\paragraph{Description:}
Reflector spillover corresponds to detector illumination by photons that do not propagate through the regular optical chain by hitting the primary, secondary, etc. This type of signal can be reduced through proper optical design procedures, but is challenging to remove entirely in a reflector system.

\paragraph{Plan to model and/or measure:}
Typically, reflector spillover is modeled through ray tracing techniques using an accurate CAD model of the telescope. Unless the spillover amplitude is high, it can be challenging to isolate and map spillover across the spatial response of the telescope. Depending on the amplitude of the spillover, it could be possible to isolate its value by observing a point source.

This effect is more significant in reflector systems and can be significant, especially in the instrument sensitivity, so it has an SRF of 4.

\paragraph{Uncertainty/Range:}
\textbf{Need some values for the SO LAT and SAC.}
\paragraph{Parameterization:}
Spillover can be quantified by the total amount of beam solid angle contained in spillover regions, $\Omega _\mathrm{spill}$ relative to the total beam solid on the sky $\Omega _\mathrm{total}$. For CMB experiments, a ratio of $\Omega _\mathrm{spill} / \Omega _\mathrm{total} > 0.01$ should be considered quite large.
