\subsection{$I\rightarrow P$ leakage and Instrumental Polarization Downstream of the HWP}
\label{IP downstream of HWP}
\paragraph{Description:}
Normal I$\rightarrow$P leakage through differential transmission will not be modulated for elements downstream of the HWP, and this sort of leakage will not contribute to the HWPSS. 
However, a small fraction of any unpolarized or polarized light that is reflected off of the detector-side of the HWP will be modulated into the 2$f$ and 4$f$ bands.
The reason for this modulation is similar to the reason unpolarized and polarized light is modulated through transmission described in section \ref{HWP Differential Optical Properties}, however here the HWP reflection mueller matrix is used instead of the transmission matrix.

The largest sources of this type of HWPSS is unpolarized light reflected off of filters and lenses, and polarized light created through differential reflection by the Alumina filters directly after the HWP. 
Though the differential reflection coefficient of the Alumina is basically the same as the differential transmission coefficient which dominates the total HWPSS, we expect this signal to be an order of magnitude smaller than the total HWPSS due to the extra reflection off of the HWP.


%Optical elements between the HWP and the detectors are sources of 2$f$ and 4$f$ signals if they emit unpolarized and polarized radiation, respectively. The reason for this effect is the HWP behavior in reflection (Sec.\,\ref{}). While it is unlikely that optical filters located after the HWP can emit a polarized radiation, the amplitude of the unpolarized one depends also on the HWP operating temperature.
%If the HWP is working at ambient temperature or at cryogenic temperatures above 4\,K there will be more optical components contributing to this effect that in the case of a HWP operating at 4\,K. The amplitude of this signal depends also on the HWP AR coating properties.



%Description of systematic effect, including relevant equations and parameterization for TWGs. Note that each variable in each equation should bedefined. This should include where we expect to get the value of this variable from (TWG, literature, etc.)


\paragraph{Plan to model and/or measure:}
The modelling of this effect is very similar to $I\rightarrow P$ leakage upstream of the HWP, described in Section \ref{IP upstream of HWP}.

An optical chain file must be provided containing the temperature, transmission, absorption, and reflection of each element. 
Differential  coefficients of Alumina Filters at varying incident angles can be computed 
using the transfer matrix method because light passing through it is in the form of a plane wave.
The HWP reflection mueller matrix is computed using Thomas Essinger Hillman's generalized tranfer matrix method \cite{Essinger-Hileman2013}.

Once we have the transmission, reflection, and absorption coefficients of each element, we are able to propagate light through the system, and calculate the unpolarized and polarized power incident on both the sky-side and detector-side of each element. 
We are then able to calculate the Stokes vector reflected off of the HWP for a range of rotation angles, multiply it by the efficiency of all elements between the HWP and the detector, and fit the first 8 harmonic amplitudes and phases.


There is no way to experimentally distinguish between the 2$f$ and 4$f$ signals due to the HWP behavior in transmission (which is generated by instrumental emission of the optical components before the HWP, the atmosphere and the incoming unpolarized background) and the HWP behavior in reflection (due to the optical elements located after the HWP). One possibility to partially measure this effect is to meausure the output detectors signal with the rotating HWP and the cryostat in dark configuration (without any incoming radiation from outside).
The 2$f$ and 4$f$ signals eventually detected come from instrumental emissions coupled to the HWP behavior in reflection and in transmission. 


%Plan to model/measure effect. Use SRFs to describe how well we understand/can model the effect. Is there a good null test that we could use to catch this effect?

\paragraph{Uncertainty/Range:}
Uncertainty resides in how well the Mueller matrix components can be experimentally measured (traditionally measurements in reflections are more complicated than the ones in transmission). The level of instrumental polarization is expected to fluctuate less in this case, with respect to the ones due to the elements before the HWP: the former resides at lower temperatures with respect to the latter. 

\textbf{Have not added most recent numbers yet....}
%This section should include the uncertainty of known parameters and/or the expected range of parameters for consideration

\paragraph{Parameterization:}
The suitable figure of merit is the amplitude of the 2$f$ and 4$f$ signals derived from the HWP in reflection coupled to the all optical elements located after the HWP.

The model of this effect requires an optical chain file containing the temperature, transmission, reflection, and absorption coefficients of each element.
These coefficients can also be calculated using the transfer matrix method, in which case the thickness and complex index of refraction is required.
%This section should include the parameterization of figures of merit and the output to the SWGs.
