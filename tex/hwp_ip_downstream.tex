\subsection{$I\rightarrow P$ leakage and Instrumental Polarization Downstream of the HWP}
\label{IP downstream of HWP}
\paragraph{Description:}
Optical elements between the HWP and the detectors are sources of 2$f$ and 4$f$ signals if they emit unpolarized and polarized radiation, respectively. The reason for this effect is the HWP behavior in reflection (Sec.\,\ref{}). While it is unlikely that optical filters located after the HWP can emit a polarized radiation, the amplitude of the unpolarized one depends also on the HWP operating temperature.
If the HWP is working at ambient temperature or at cryogenic temperatures above 4\,K there will be more optical components contributing to this effect that in the case of a HWP operating at 4\,K. The amplitude of this signal depends also on the HWP AR coating properties.

%Description of systematic effect, including relevant equations and parameterization for TWGs. Note that each variable in each equation should bedefined. This should include where we expect to get the value of this variable from (TWG, literature, etc.)
Say something about how the I to P of elements downstream of the HWP are not modulated so this I to P is mitigated by the HWP. However, some polarized reflections off of these elements can reflect off of the HWP and cause I to P leakage.

\paragraph{Plan to model and/or measure:}
This effect can be estimated combining together the radiation emission laws for the optical components (e.g. black bodies with some estimated emissivity) with the Mueller matrix of the HWP (plus AR coating) in reflection. 

There is no way to experimentally distinguish between the 2$f$ and 4$f$ signals due to the HWP behavior in transmission (which is generated by instrumental emission of the optical components before the HWP, the atmosphere and the incoming unpolarized background) and the HWP behavior in reflection (due to the optical elements located after the HWP). One possibility to partially measure this effect is to meausure the output detectors signal with the rotating HWP and the cryostat in dark configuration (without any incoming radiation from outside).
The 2$f$ and 4$f$ signals eventually detected come from instrumental emissions coupled to the HWP behavior in reflection and in transmission. 


%Plan to model/measure effect. Use SRFs to describe how well we understand/can model the effect. Is there a good null test that we could use to catch this effect?

\paragraph{Uncertainty/Range:}
Uncertainty resides in how well the Mueller matrix components can be experimentally measured (traditionally measurements in reflections are more complicated than the ones in transmission). The level of instrumental polarization is expected to fluctuate less in this case, with respect to the ones due to the elements before the HWP: the former resides at lower temperatures with respect to the latter. 
%This section should include the uncertainty of known parameters and/or the expected range of parameters for consideration

\paragraph{Parameterization:}
The suitable figure of merit is the amplitude of the 2$f$ and 4$f$ signals derived from the HWP in reflection coupled to the all optical elements located after the HWP.
%This section should include the parameterization of figures of merit and the output to the SWGs.
