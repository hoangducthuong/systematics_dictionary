\subsection{Atmospheric gain correction}

\paragraph{Description:}
The scale of the atmosphere fluctuations follow a power law dominated by large scales, which are the dominant signal on the detectors, producing strong correlations between them. We can use these correlations to obtain a correction to the calibration mismatch. 

On top of this, the gain corrections suffer from atmospheric substructure. Sub-array atmosphere modes contaminate the calibration by leaking power into the common mode estimator (mean or median), biased towards the sub-array modes which are better represented or correlated between many detectors. We expect this effect to generate higher common mode correlations in the center of the array, as those sub-array atmosphere modes will be better sampled compared to those on the edges.

The procedure used for ACT is the following:

\begin{enumerate}
\item Fourier transform the data and select a working frequency band to do the analysis. In ACT we often select frequencies between 0.017 to 0.088 Hz. The following steps are performed on this subset of the data in Fourier space.
\item Apply the electrical responsivity, computed using IV or BiasStep tests, and a fixed optical flatfield computed by other means.
\item Deproject the dark detector signal from the working (live) detectors data, in order to extract the thermal contamination. One or more dark TODs are fitted and subtracted sequentially.
\item Compute the common mode as the per sample mean (or median) of all ``well behaved`` live detector TODs.
\item Fit the common mode to each detector TOD, such that the fit coefficients represent the gain of individual detectors referred to the common mode.
\item Recover the overall gain level by normalizing by the mean of the gain of a subset of detectors considered stable, meaning that we trust their electrical responsivity under different environmental conditions.
\end{enumerate}

The stability of this calibration over long periods of time is dominated by the right selection of stable detectors. These detectors have been selected for having reliable bias-step responsibities over a wide range of atmospheric conditions, but we lack of a more systematic method to select the stable detectors. 

\paragraph{uncertainty/range:}
The uncertainty in the atmosphere gain corrections in now limited by our ability to measure it. In the case of ACT, that con only be done using planet observations, requiring single detector fits to the planet signal, with an uncertainty between 8 to 10\%. We expect the precision of the correction to be better than that at the 150 GHz band. On the other hand, we have seen that the technique is not useful at low frequency bands. In particular, we have shown that the relative calibration worsen when applied to 90 GHz data.


\paragraph{Plan to measure:}
We would like to know the fraction of the time when this effect is relevant and if it is correlated with the optical loading or amplitude of the atmospheric fluctuations. For this reason we plan to study the PWV dependency of the technique. Plot correlation vs loading. Plot correlation vs TOD norm

\paragraph{Parameterization:}
