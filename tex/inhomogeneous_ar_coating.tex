\subsection{Inhomogeneous AR Coating}

\paragraph{Description:}
If a telescope has any transmissive opitcal elements, those elements will need to have an effective antireflective (AR) coating.  Some systematics are particular to the AR technology used, but there are some general systematics.  Inhomogeneities can occur with most technologies, such as by a variation in the thickness of a layer across the element.  This will cause a decrease in the AR performance at the location of the inhomogeneity.  If the element is in a position in the beam such that all the detectors effectively see the entire element, this will be averaged over the entire element, and the systematic will be the same across the whole focal plane.  If the element is in a position such that each detector sees only a small part of the element, there will be a focal plane positional dependence on the transmission.  If there are several such elements in the optical path for each detector, the effect will hopefully average out over the focal plane.

\paragraph{Plan to model and/or measure:}
The AR coating performance of each element can be measured with a reflectometer. Measurements of specific technologies can give the tolerances of each technology. Using reflectometery, we can measure reflection vs. frequency data for these designs, understand the thickness and index tolerances of any proposed AR technology, and determine an rms for each layer of AR coating. Using the tolerances, rms values, and a transfer matrix model can give predictions on how much this will affect the overall AR coating performance. Given the reasonably tight tolerances of current technologies, this will not be a major problem and thus has a SRF of 2.

\paragraph{Uncertainty/Range:}
Most technologies can get to a paricular thickness consistency across the surface. This may be from $\pm$ 5 to 25 $\mu$m depending on the technology. Overall, this is a relatively small amount at most frequencies though at the upper end of ground-based range, it may start to present a problem. If the AR performance is tightly tuned across the band (to about 0.5\% or less across the band), then any deviation will cause the the performance to degrade to $\sim 1$\%. If the performance is closer to $\sim 1$\% or the coating has a very broad bandwidth, then this effect will shift the performace bands around by a few percent.  

\paragraph{Parameterization:}
The transfer matrix formalism with the measured tolerances and rms can give the total impact on the AR coating performance.
