% !TEX root =  ../syst_master.tex 

\subsection{Cross Polarization}

\paragraph{Description:}
Cross polarization is an optical systematic that shows how much polarization leakage there is between orthogonal polarization modes. Typically it is a characteristic of the optical design itself and represents how much polarization is rotated as it propagates through the optics. Alternatively, it can come from differencing detectors with different beam shapes. Cross polarization decreases polarization efficiency. If not properly accounted for, it can cause Q to U leakage, which causes E-modes to leak into B-modes.
 
It can be modeled using the Mueller matrix formalism. If the telescope Mueller beam matrix is known, these systematics (along with beam effects) can be propagated to the Q, U maps by
\begin{equation}
Q' = m_{qq} Q + m_{qu} U, \ \ U' = m_{uu} U + m_{uq} Q
\end{equation}
In this way, the Q and U maps with this systematic included can be simulated. The contaminated Q and U maps can be further propagated to the power spectra determine the level of E-mode to B-mode leakage.

For the feedhorn and lenslet cross polarization modeling, the cross polarization is determined from the polarized ($rEL3X$ and $rEL3Y$) beam parameters in dB from HFSS. The E-plane beam is given by $rEL3X$ at $\phi=0^{\circ}$, where $\phi$ is the angular coordinate of the beam, and the H-plane beam is given by $rEL3X$ at $\phi=90^{\circ}$. The cross polarization beam is given by $rEL3Y$ at $\phi=45^{\circ}$. All beams are normalized by the maximum value of the E-plane beam, such that the maximum value of the E-plane beam is equal to one. The cross polarization is then the maximum value of the cross polarization beam within the aperture stop.

\paragraph{Plan to model and/or measure:}
This effect can be modeled for the telescope optics using physical optics simulations where the Mueller matrix can be calculated directly. For the feedhorn and lenslets, the crosspolarization can be modeled using HFSS. Cross polarization will be measured and calibrated out with polarization angle calibration using a both a polarized source or wire grid. We will compare the measured cross polarization with the modeled system cross polarization. The feedhorn/lenslet cross polarization can be modeled and measured separately from the full system to differentiate how much cross polarization originates from the telescope versus the feedhorns/lenslets. For feedhorns, the beams will be measured at room-temperature using a VNA and/or holography beam mapping system to measure cross polarization and compare to HFSS simulations. For lenslets, this will be achieved through measuring the cold beams of lenslets and comparing to HFSS models.

Cross polarization is small and can be calibrated out with polarization angle measurements, but it must be modeled. It thus has a SRF of 3.

\paragraph{Uncertainty/Range:}
The typical allowable ranges for this value are usually $<-15$~dB to $<-30$~dB in power, but this needs to be further constrained by the SWG for this specific project. For comparison, the cross polarization for the AdvACT 90/150~GHz feedhorns is $1.74\%$ in the low band and $0.3\%$ in the high band, while the values for the AdvACT 150/230~GHz feedhorns are $1\%$ in the low band and $0.4\%$ in the high band (though we note that these values take only the maximum of the cross-polar beam and do not account for the aperture stop). \textbf{To reach a negligible level of xx\%, we require a polarization angle calibration to XX$^{\circ}$.}

\paragraph{Parameterization:}
We will parameterize this effect with the contaminuated Q and U maps for the telescope and the cross-polar beams from HFSS for the feedhorns/lenslets (see more details on this in the feedhorn/lenslet polarization leakage section).
