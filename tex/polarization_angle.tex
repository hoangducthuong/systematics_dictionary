\subsection{Absolute Polarization Angle}

\paragraph{Description:}
Absolute Polarization Orientation refers to the polarimeter detectors' direction measured in celestial coordinates. A miscalibration (i.e. a rotation bias for the detector orientation) mixes E-modes and B-modes. In addition to contaminating the CMB polarization power spectra, such a systematic rotation is degenerate with Cosmic Birefringence (CB) and Cosmic Polarization Rotation (CPR).

Sources of polarization angle systematics are varied and can be introduced
several places in the instrument. A few examples include 1) a rotating elliptical beam, say
in the case of a design incorporating bore-sight rotations, causing T to P leakage (see ellipticity section); 2) off-axis refractive optics influencing the propagation of the polarization vectors according to their Fresnel coefficients, leading to an instrumental polarization angle rotation; and 3) an apparent polarization angle rotation from the detector time constants in the presence of a HWP (see time constant section). Here we focus on 2), namely instrumental polarization errors and
detector polarization angle rotations.

A global polarization rotation is degenerate with a CPR angle and affects the
power spectra as described in \cite{2013ApJ...762L..23K}. Analytic description
of instrumental rotation is challenging, necessitating the use of optical
modeling and experimental techniques for calibration of final detector angles
(absolute and relative) and systematic rotations from the optics.

\paragraph{Plan to model and/or measure:}

It is critical to both model and measure the detector polarization angles.
Calibration should be performed before deployment and during observations. 
We plan to use polarization calibrators in-lab while testing full optics tubes before deployment, in situ in the field with polarization calibrators (either drone/ground-based), and observations polarized astrophysical sources like Tau A (the Crab
Nebula) and Cen A \textbf{give some citations}.

When considering a lab source, placing a well known polarization
calibrator in the far field is preferred, though difficult in practice.
Proposed ideas include flying a source (whether tunable or wide band) on a
drone or on a CalSat to place it in the far field. Alternative calibrators
require placement in the near field and include sparse/dense wire grid
polarizers or dielectric sheets \cite{Takahashi2010, 2016arXiv160701825K}. \textbf{Say something about the PB calibrator that Grant has been working on, something about PoloCalc, and other options. Note that it would be easier to measure in-lab if we only have the optics tubes and what options we have for that. See the calibration hardware spreadsheet linked on the CSS telecon page.}

Modeling of the polarization rotation angle appears feasible and has been used
on ACTPol using Code V \cite{2016arXiv160701825K}. This should be checked with
physical optics calculations and measurements, but can be performed on a proposed telescope
design. \textbf{Add a few more details about the modeling--one of Brian Koopman's plots?}

\textbf{Can we use EB and TB cross-crrelations to help in our null tests for absolute polarization angle?}

Understanding the polarization anlge is critical to achieving several science goals and thus has an SRF of 5.

\paragraph{Uncertainty/Range:}
%This section should include the uncertainty of
%known parameters and/or the expected range of parameters for consideration

Angle offsets $\sim 1^{\circ}$ produce spurious B-mode signal at the same level
as primodial B-modes for a tensor to scalar ratio of $r \sim 0.005$ as well as
nonzero $EB$ and $TB$ cross-correlations \cite{doi:10.1142/S0218271816400125}. \textbf{Say what the r goal for SO is, and what angle calibration that requires.}
Currently employed calibration methods provide calibration to at best
$0.5^{\circ}$ \cite{2016MNRAS.455.1981K} \textbf{Is this sufficient for SO's science goals or do we need more R and D to reach our scienc goals?}. Calibration to better than
$0.05^{\circ}$ would allow for constraints on CPR of order one degree to
greater than $15\sigma$ \cite{2016MNRAS.455.1981K} \textbf{What about on r?}.  \textbf{Use some of the plots and numbers from Federico's paper here}

\paragraph{Parameterization:}
This effect can be parameterized by the polarization angle uncertainty, which can be used to estimate the polarization power spectra leakages as in \textbf{cite Federico's paper}. This can be used to estimate the impact on our science goals.
