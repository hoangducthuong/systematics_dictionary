\subsection{Thermal Variation in HWP Temperature}

\paragraph{Description:}
In practice, the HWP has thermal gradients and fluctuations. A non-uniform HWP temperature and its temporal fluctuations create spatial and time dependent 0$f$, 1$f$, 2$f$, and 4$f$ signals that depend on the amplitude of the harmonic decomposition of the temperature profile of the HWP. 

\textbf{Add a few sentences about how estimating the HWP template wrong because it is spatially and temporally varying can translate into leakage.}

\paragraph{Plan to model and/or measure:}
To model this effect, we can use a thermal finite element analysis of a HWP thermalized with a suitable heat sink and with realistic time variable loading. \textbf{How can we estimate the next step (the impact on leakage from this)?}
To measure this, we can use cryogenic measurements of the thermal gradients, and on-sky measurements of polarization leakage will include this effect (as well as other instrumental sources)

This effect is expected to be subdominant, so the SRF is 3.


\paragraph{Uncertainty/Range:}
The uncertainties in these models will be limited by how realistic models of the time variation of the loading are. For a sapphire HWP, we expect the time variation effect to be small, as sapphire has a large thermal conductivity and large heat capacity, giving it both good thermalization properties and a large time constant.
 \textbf{Add more detail if we mdoel this}

\paragraph{Parameterization:}
We can parameterize this using the HWP thermal fluctuations: $\Delta_{HWP}(x,y,t)$ with $(x,y)$ the spatial coordinates of the HWP and $t$ the time and the effect on the HWP thermal fluctuation time constant: 

\begin{equation}
\tau_{T_{HWP}} = C(T_{HWP})/G(T_{HWP}, T_{BCK})
\end{equation}

where $C$ is the HWP heat capacity as a function of its temperatue $T_{HWP}$, and $G$ is the thermal link between the HWP at $T_{HWP}$ and the surrounding environment at $T_{BCK}$. 

Using this, we expect the HWP temperature to fluctuate, assuming a fluctuating background at frequency $\omega_{BCK}$, as

 \begin{equation} 
 T_{HWP} \propto \frac{e^{i \omega_{BCK} t}}{i \omega_{BCK} + 1/\tau_{T_{HWP}}}
 \end{equation}
 
 Because the time constant is large, we expect the thermal fluctuation to manifest at very low frequencies $\omega_{BCK}$. To estimate the scientific impact, we could parameterize this with spatially and temporally varying HWP template values.
