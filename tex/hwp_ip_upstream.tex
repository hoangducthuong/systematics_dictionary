% !TEX root =  ../syst_master.tex 

\subsection{$I\rightarrow P$ leakage and Instrumental Polarization upstream of the HWP}
\label{IP upstream of HWP}

\paragraph{Description:}
Any polarized signal originating from the telescope upstream of the HWP will be rotated and coupled to the detector at 4f,
making it indistinguishable from the sky polarization.
This includes the intensity dependent $I\rightarrow P$ leakage, 
described and calculated in section \ref{instrumental_polarization} 
along with polarized emission from optical elements. 

The IP coefficient and polarized emissivity are both caused by different transmission coefficients along orthogonal axes,
and are equal in magnitude but opposite in sign.
They interfere destructively but do not completely cancel, since the modified black body spectrum (Eq. \ref{}) is at the temperature
of the optical element, which is not the same as the spectrum of the source.

Because the majority of optical elements are operated at low temperatures, 
the power of the polarized emission is usually small compared to that of the $I\rightarrow P$ leakage, and so they are ignored.
The exception to this the mirrors which are kept at 300 K, and so the mirror polarized emission must be included in our prediction.

\paragraph{Plan to model and/or measure:}
In order to predict the 4f signal generated by an optical element, 
we need to know the unpolarized power incident on the element ($P^u_n(\nu)$), the IP coefficient of the element 
($\eta_n^\text{IP}$), the location of the HWP,
and the combined polarized efficiency of everything between that element and the detector ($\eta_n^\text{det}$).

To get the unpolarized power incident on an optical element we require an optical chain data.
The optical chain file must contain details such as the absorption, spill, scatter, reflection coefficients, and 
the temperature for each element in the chain.
Using this, we can propagate unpolarized power via the method described in section \ref{HWP Differential Optical Properties}. 
An example of an optical chain file with different possible HWP positions is given in this 
\href{http://simonsobservatory.wdfiles.com/local--files/calandsys-telecon/eb_leakage_from_pointing_error.pdf?ukey=61f26ef33e8439a4e7096ab52c54c523066a4e35}{memo}.

The calculation of major IP coefficients is described in section \ref{instrumental_polarization}.


The calculation of the polarized power output of an optical element is similat to what has been described in equation (\ref{eq:polarized_output}).
Since the polarized emissivity and IP of an optical element are equal, the polarized power is given by:
\begin{equation}
P^p_{n+1}(\nu) = \eta^\text{IP}_n \left(P_n^u (\nu) - A\Omega(\nu) B(\nu, T_n) \right).
\end{equation}
The total polarized power seen by the detector by an optical element is then
\begin{equation}
P^p_{n, \text{total}} = \left|\int_{\nu_\text{low}}^{\nu_\text{high}} \eta_n^\text{det}(\nu) P^p_n(\nu) d\nu\right|.
\end{equation}
We take the absolute value to ensure that polarized signal from different optical elements will not interfere
with each other.
A more detailed study will take into account the polarization angle of the light and add them accordingly.

This calculation is done for all elements upstream of the HWP.

\paragraph{Uncertainty/Range:}
Using this model for 145 GHz, we can expect values of $A^{(4)}$ in the range of 0.027 pW which converts to 0.399 K$_\text{CMB}$ using the optical chain mentioned in the memo. This increases slightly if the HWP is positioned later in the optical chain 
because more lenses are taken into account.

It is interesting to note that the majority of this signal ($\sim 87\%$) comes from the polarized emission of the mirrors, 
since they are kept at room temperature.

\paragraph{Parameterization:}
For this calculation we need to know the properties of the optical chain (e.g. absorption, spill, scatter, reflection coefficients, and 
the temperature of each element chain). We also need IP values for lenses and windows and the location of the HWP. 


