\subsection{Instrumental Polarization} 
\label{instrumental_polarization}

Instrumental polarization (IP) is an optical systematic that shows how much intensity signal is leaking into the polarization signal. It can come from the optics (depends on the properties of metals and dielectric materials), or come from pair differencing with detectors having different beams. For optical elements, differential transmission along orthogonal axes will polarize incident unpolarized light. The optical components also emit polarized light for the same reason, but with opposite polarization angle. \textbf{What is the difference between an optical component and an optical element here?}
This systematic will leak the temperature signal into E-modes and B-modes, resulting in large contaminations if not properly accounted for in analysis. Leakage as a result of pair differencing is discussed in more detail in the ellipticity, differntial pointing, and polarization leakage from feedhorns/lenslets sections.

It can be modeled using the Mueller matrix formalism. The effect is worse at the edges of the focal plane because of the increased incident angles on the optical elements. If the telescope Mueller beam matrix is known, these systematics (along with beam effects) can be propagated to the Q and U maps by
\begin{equation}
Q' = m_{qq} Q + m_{qi} I, \ \ U' = m_{uu} U + m_{ui} I \, .
\end{equation}
In this way, we can simulate Q and U maps with the systematic contamination. The contaminated Q and U maps can be further propagated to the power spectra to determine the magnitude of the polarization leakage.

\paragraph{Plan to model and/or measure:} \mbox{}\\

This effect can be modeled using physical optics simulations where the Mueller matrix can be calculated directly. 
The main optical elements to consider are mirrors, lenses, and filters.
A study of IP of filters can be found in \cite{pisano2005}.

The combined IP of the windows and lenses for ACTPol was calculated using Code V, 
by putting an unpolarized input on the sky and propagating it to the focal plane.
The IP is larger towards the edges of the detector array due to the non-zero incident angles.
This gives values of $\sim0.12\%$ and $\sim0.015\%$ at the edges and the center respectively.
For now we assume that this IP is divided equally among the lenses. \textbf{Are there updated values for the LAT and SA designs?}

Optical leakage also occurs at the mirrors due to their finite conductivity. 
The formalism is presented in \cite{Barkats:2005sh} and is given by the Hagen-Rubens formula multiplied by 
a geometric factor determined by the incident angle:
\begin{equation}
\lambda_\text{opt}(\nu) = \sqrt{4 \pi \epsilon_0 \nu \rho} (\sec \chi - \cos \chi) \, ,
\end{equation}
where $\rho$ is the conductivity of the mirror and $\chi$ the incident angle.

Currently we use the mirror specifications of CCAT, which gives incident angles of $25.73^\circ$ for the primary mirror 
and $19.59^\circ$ for the secondary. The IP of both mirrors together ends up around $0.04\%$ at 145 degrees. \textbf{Are there updated values for the LAT and SA designs?}

In the case of a HWP, only the IP up to the HWP needs to be considered.
A more detailed discussion of how to calculate the power coming from IP and instrumental polarization
seen by the detector is given in section \ref{HWP Differential Optical Properties}.
A first attempt at tallying up the IP from the optical chain in SO is available in this \href{http://simonsobservatory.wdfiles.com/local--files/calandsys-telecon/eb_leakage_from_pointing_error.pdf?ukey=61f26ef33e8439a4e7096ab52c54c523066a4e35}{memo}. \textbf{Link updated memos for the LAT and SAC and give a brief summary of the results in this text. Joy and Jack likely have the most up to date numbers.}

\textbf{Can we measure individual elsements with a reflectometer/holography system and compare to simulations? Say more about how we can characterize and remove this with measurements/calibration.}

This effect is difficult to model and has a direct impact on the achievable science. Its SRF is thus a 4.

\paragraph{Uncertainty/Range:}
Using the methods above for the lenses and mirrors, we see an IP of about $0.16\%$ for the whole system.
In a more detailed study, we will integrate over all incident angles over each of the optical elements and add up the IP of various elements coherently. \textbf{Are these numbers up to date? Have we done the more detailed study?}

\paragraph{Parameterization:}
We parameterize this effect with the instrument Mueller matrix elements $m_{qi}$ and $m_{ui}$ as a function of frequency and incident angle. We use these to make Q and U maps that can be used to estimate the power spectrum leakage and thus the impact on our science objectives.
