\subsection{Instrumental Polarization} 
\label{instrumental_polarization}

Instrumental polarization (IP) is an optical systematic that shows how much intensity signal is leaking into the polarization signal. It can come from the optics (depends on the properties of metals and dielectric materials), or come from pair differencing with detectors having different beams. Any differential transmission or reflection coefficients along polarization axes will polarize incident unpolarized light. 
Differential absorption coefficients will cause emitted light to be partially polarized. This effect is not proportional to incoming intensity, but the polarization is especially strong for warm elements such as mirrors.
This systematic will leak the temperature signal into E-modes and B-modes, resulting in large contaminations if not properly accounted for in analysis. Leakage as a result of pair differencing is discussed in more detail in the ellipticity, differential pointing, and polarization leakage from feedhorns/lenslets sections.

It can be modeled using the Mueller matrix formalism. The effect is worse at the edges of the focal plane because of the increased incident angles on the optical elements. If the telescope Mueller beam matrix is known, these systematics (along with beam effects) can be propagated to the Q and U maps by
\begin{equation}
Q' = m_{qq} Q + m_{qi} I, \ \ U' = m_{uu} U + m_{ui} I \, .
\end{equation}
In this way, we can simulate Q and U maps with the systematic contamination. The contaminated Q and U maps can be further propagated to the power spectra to determine the magnitude of the polarization leakage.

\paragraph{Plan to model and/or measure:} \mbox{}\\

This effect can be modeled using physical optics simulations where the Mueller matrix can be calculated directly. 
The main optical elements to consider are mirrors, lenses, and filters.
A study of IP of filters can be found in \cite{pisano2005}.

The combined IP of the windows and lenses for ACTPol was calculated using Code V, 
by putting an unpolarized input on the sky and propagating it to the focal plane.
The IP is larger towards the edges of the detector array due to the non-zero incident angles.
This gives values of $\sim0.12\%$ and $\sim0.015\%$ at the edges and the center respectively.
For now we assume that this IP is divided equally among the lenses. 




Optical leakage also occurs at the mirrors due to their finite conductivity. 
The formalism is presented in \cite{Barkats:2005sh} and is given by the Hagen-Rubens formula multiplied by 
a geometric factor determined by the incident angle:
\begin{equation}
\lambda_\text{opt}(\nu) = \sqrt{4 \pi \epsilon_0 \nu \rho} (\sec \chi - \cos \chi) \, ,
\end{equation}
where $\rho$ is the conductivity of the mirror and $\chi$ the incident angle.


For the SAT, we only must consider IP that is modulated by the HWP, which primarily comes from elements upstream of the HWP.
The differential coefficients for each element are calculated using the transfer matrix method.
The IP is dominated by the differential transmission and reflection of Alumina filters, both of which are $\sim.46\%$ for a $20^\circ$ incidence angle. 
The window also has differential transmission and reflection coefficients, but this is much smaller,
at $\sim.09\%$ for a $20^\circ$ incidence angle.
A detailed summary of how we calculate the IP for the SAT is given in sections \ref{HWP Differential Optical Properties},
\ref{IP upstream of HWP} and \ref{IP downstream of HWP}. 

To help understand the source of and mitigate this leakage, we can measure individual elements with a reflectometry system at different angles of incidence and compare the results to the simulations. Additionally, full optics tubes can be placed in a beam mapping setup to determine if the polarization leakage of the full optical system is as expected. The total polarization leakage of the telescope can be measured in the field by observing unpolarized sources and determining the polarized signal. These sources could be tower or drone mounted or celestial.

This effect is difficult to model and has a direct impact on the achievable science. Its SRF is thus a 4.

\paragraph{Uncertainty/Range:}
Using the methods above for the lenses and mirrors, we see an IP of about $0.16\%$ for the LAT.
In a more detailed study, we will integrate over all incident angles over each of the optical elements and add up the IP of various elements coherently. \textbf{Are these numbers up to date? Have we done the more detailed study?}

For the Small Aperture Telescope, after adding IP from all elements and taking into account the HWP modulation efficiency,
the percent of unpolarized power ending up in the 2$f$ and 4$f$ bands at 145 GHz is given by $a_2 = 0.331\%$ and $a_4 = 0.326\%$ 
at the start of the telescope for a $20^\circ$ incidence angle.


Currently we use the mirror specifications of CCAT, which gives incident angles of $25.73^\circ$ for the primary mirror 
and $19.59^\circ$ for the secondary. The IP of both mirrors together ends up around $0.04\%$ at 145 degrees. \textbf{Add updated values for the LAT and SAC designs once available.}

\paragraph{Parameterization:}
We parameterize this effect with the instrument Mueller matrix elements $m_{qi}$ and $m_{ui}$ as a function of frequency and incident angle. We use these to make Q and U maps that can be used to estimate the power spectrum leakage and thus the impact on our science objectives.
